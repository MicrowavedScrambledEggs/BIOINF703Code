\documentclass[]{article}
\usepackage{lmodern}
\usepackage{amssymb,amsmath}
\usepackage{ifxetex,ifluatex}
\usepackage{fixltx2e} % provides \textsubscript
\ifnum 0\ifxetex 1\fi\ifluatex 1\fi=0 % if pdftex
  \usepackage[T1]{fontenc}
  \usepackage[utf8]{inputenc}
\else % if luatex or xelatex
  \ifxetex
    \usepackage{mathspec}
  \else
    \usepackage{fontspec}
  \fi
  \defaultfontfeatures{Ligatures=TeX,Scale=MatchLowercase}
\fi
% use upquote if available, for straight quotes in verbatim environments
\IfFileExists{upquote.sty}{\usepackage{upquote}}{}
% use microtype if available
\IfFileExists{microtype.sty}{%
\usepackage{microtype}
\UseMicrotypeSet[protrusion]{basicmath} % disable protrusion for tt fonts
}{}
\usepackage[margin=1in]{geometry}
\usepackage{hyperref}
\hypersetup{unicode=true,
            pdftitle={BIOINF703 Lab 1},
            pdfauthor={Badi James},
            pdfborder={0 0 0},
            breaklinks=true}
\urlstyle{same}  % don't use monospace font for urls
\usepackage{color}
\usepackage{fancyvrb}
\newcommand{\VerbBar}{|}
\newcommand{\VERB}{\Verb[commandchars=\\\{\}]}
\DefineVerbatimEnvironment{Highlighting}{Verbatim}{commandchars=\\\{\}}
% Add ',fontsize=\small' for more characters per line
\usepackage{framed}
\definecolor{shadecolor}{RGB}{248,248,248}
\newenvironment{Shaded}{\begin{snugshade}}{\end{snugshade}}
\newcommand{\KeywordTok}[1]{\textcolor[rgb]{0.13,0.29,0.53}{\textbf{#1}}}
\newcommand{\DataTypeTok}[1]{\textcolor[rgb]{0.13,0.29,0.53}{#1}}
\newcommand{\DecValTok}[1]{\textcolor[rgb]{0.00,0.00,0.81}{#1}}
\newcommand{\BaseNTok}[1]{\textcolor[rgb]{0.00,0.00,0.81}{#1}}
\newcommand{\FloatTok}[1]{\textcolor[rgb]{0.00,0.00,0.81}{#1}}
\newcommand{\ConstantTok}[1]{\textcolor[rgb]{0.00,0.00,0.00}{#1}}
\newcommand{\CharTok}[1]{\textcolor[rgb]{0.31,0.60,0.02}{#1}}
\newcommand{\SpecialCharTok}[1]{\textcolor[rgb]{0.00,0.00,0.00}{#1}}
\newcommand{\StringTok}[1]{\textcolor[rgb]{0.31,0.60,0.02}{#1}}
\newcommand{\VerbatimStringTok}[1]{\textcolor[rgb]{0.31,0.60,0.02}{#1}}
\newcommand{\SpecialStringTok}[1]{\textcolor[rgb]{0.31,0.60,0.02}{#1}}
\newcommand{\ImportTok}[1]{#1}
\newcommand{\CommentTok}[1]{\textcolor[rgb]{0.56,0.35,0.01}{\textit{#1}}}
\newcommand{\DocumentationTok}[1]{\textcolor[rgb]{0.56,0.35,0.01}{\textbf{\textit{#1}}}}
\newcommand{\AnnotationTok}[1]{\textcolor[rgb]{0.56,0.35,0.01}{\textbf{\textit{#1}}}}
\newcommand{\CommentVarTok}[1]{\textcolor[rgb]{0.56,0.35,0.01}{\textbf{\textit{#1}}}}
\newcommand{\OtherTok}[1]{\textcolor[rgb]{0.56,0.35,0.01}{#1}}
\newcommand{\FunctionTok}[1]{\textcolor[rgb]{0.00,0.00,0.00}{#1}}
\newcommand{\VariableTok}[1]{\textcolor[rgb]{0.00,0.00,0.00}{#1}}
\newcommand{\ControlFlowTok}[1]{\textcolor[rgb]{0.13,0.29,0.53}{\textbf{#1}}}
\newcommand{\OperatorTok}[1]{\textcolor[rgb]{0.81,0.36,0.00}{\textbf{#1}}}
\newcommand{\BuiltInTok}[1]{#1}
\newcommand{\ExtensionTok}[1]{#1}
\newcommand{\PreprocessorTok}[1]{\textcolor[rgb]{0.56,0.35,0.01}{\textit{#1}}}
\newcommand{\AttributeTok}[1]{\textcolor[rgb]{0.77,0.63,0.00}{#1}}
\newcommand{\RegionMarkerTok}[1]{#1}
\newcommand{\InformationTok}[1]{\textcolor[rgb]{0.56,0.35,0.01}{\textbf{\textit{#1}}}}
\newcommand{\WarningTok}[1]{\textcolor[rgb]{0.56,0.35,0.01}{\textbf{\textit{#1}}}}
\newcommand{\AlertTok}[1]{\textcolor[rgb]{0.94,0.16,0.16}{#1}}
\newcommand{\ErrorTok}[1]{\textcolor[rgb]{0.64,0.00,0.00}{\textbf{#1}}}
\newcommand{\NormalTok}[1]{#1}
\usepackage{graphicx,grffile}
\makeatletter
\def\maxwidth{\ifdim\Gin@nat@width>\linewidth\linewidth\else\Gin@nat@width\fi}
\def\maxheight{\ifdim\Gin@nat@height>\textheight\textheight\else\Gin@nat@height\fi}
\makeatother
% Scale images if necessary, so that they will not overflow the page
% margins by default, and it is still possible to overwrite the defaults
% using explicit options in \includegraphics[width, height, ...]{}
\setkeys{Gin}{width=\maxwidth,height=\maxheight,keepaspectratio}
\IfFileExists{parskip.sty}{%
\usepackage{parskip}
}{% else
\setlength{\parindent}{0pt}
\setlength{\parskip}{6pt plus 2pt minus 1pt}
}
\setlength{\emergencystretch}{3em}  % prevent overfull lines
\providecommand{\tightlist}{%
  \setlength{\itemsep}{0pt}\setlength{\parskip}{0pt}}
\setcounter{secnumdepth}{0}
% Redefines (sub)paragraphs to behave more like sections
\ifx\paragraph\undefined\else
\let\oldparagraph\paragraph
\renewcommand{\paragraph}[1]{\oldparagraph{#1}\mbox{}}
\fi
\ifx\subparagraph\undefined\else
\let\oldsubparagraph\subparagraph
\renewcommand{\subparagraph}[1]{\oldsubparagraph{#1}\mbox{}}
\fi

%%% Use protect on footnotes to avoid problems with footnotes in titles
\let\rmarkdownfootnote\footnote%
\def\footnote{\protect\rmarkdownfootnote}

%%% Change title format to be more compact
\usepackage{titling}

% Create subtitle command for use in maketitle
\newcommand{\subtitle}[1]{
  \posttitle{
    \begin{center}\large#1\end{center}
    }
}

\setlength{\droptitle}{-2em}

  \title{BIOINF703 Lab 1}
    \pretitle{\vspace{\droptitle}\centering\huge}
  \posttitle{\par}
    \author{Badi James}
    \preauthor{\centering\large\emph}
  \postauthor{\par}
      \predate{\centering\large\emph}
  \postdate{\par}
    \date{August 7, 2018}


\begin{document}
\maketitle

\subsection{Plotting G}\label{plotting-g}

\begin{Shaded}
\begin{Highlighting}[]
\NormalTok{vTidy <-}\StringTok{ }\KeywordTok{sapply}\NormalTok{(vNames, }\ControlFlowTok{function}\NormalTok{(x) }\KeywordTok{unlist}\NormalTok{(}\KeywordTok{strsplit}\NormalTok{(x, }\StringTok{"_"}\NormalTok{))[}\DecValTok{1}\NormalTok{])}

\KeywordTok{plot.network}\NormalTok{(G, }\DataTypeTok{mode =} \StringTok{"kamadakawai"}\NormalTok{, }\DataTypeTok{edge.col =} \KeywordTok{rgb}\NormalTok{(}\DataTypeTok{red =} \DecValTok{0}\NormalTok{, }\DataTypeTok{blue =} \DecValTok{0}\NormalTok{, }\DataTypeTok{green =} \DecValTok{0}\NormalTok{, }\DataTypeTok{alpha =} \FloatTok{0.05}\NormalTok{), }\DataTypeTok{label =}\NormalTok{ vTidy, }\DataTypeTok{label.cex =} \FloatTok{0.5}\NormalTok{, }\DataTypeTok{label.col =} \StringTok{"blue"}\NormalTok{)}
\end{Highlighting}
\end{Shaded}

\includegraphics{Lab2_files/figure-latex/plottingG-1.pdf}

This visualization of the network shows with some clarity the conections
between the vertexes on the outside of the plot. However for the center
of the plot the edges are indistinguishable. This does provide an
overall picture of the complexity of the network but does not provide
usefull information on the specific connections. A better approach would
be to seperate the network into subnetworks. As this is a gene
interaction network, the graph could be divided by different known
cellular signalling pathways.

\subsection{Plotting degree}\label{plotting-degree}

\begin{Shaded}
\begin{Highlighting}[]
\NormalTok{gNodeDegs <-}\StringTok{ }\KeywordTok{degree}\NormalTok{(G, }\DataTypeTok{gmode =} \StringTok{"graph"}\NormalTok{)}
\KeywordTok{hist}\NormalTok{(gNodeDegs, }\DataTypeTok{ylim =} \KeywordTok{c}\NormalTok{(}\DecValTok{0}\NormalTok{,}\DecValTok{35}\NormalTok{), }\DataTypeTok{xlim =} \KeywordTok{c}\NormalTok{(}\DecValTok{0}\NormalTok{,}\DecValTok{100}\NormalTok{), }\DataTypeTok{xlab =} \StringTok{"Degree"}\NormalTok{, }\DataTypeTok{main =} \StringTok{"Degree distribution of G"}\NormalTok{)}
\end{Highlighting}
\end{Shaded}

\includegraphics{Lab2_files/figure-latex/degree-1.pdf}

\subsection{Binomial (Simulating an Erdos-Renyi model for
G)}\label{binomial-simulating-an-erdos-renyi-model-for-g}

\begin{Shaded}
\begin{Highlighting}[]
\NormalTok{gDegMean <-}\StringTok{ }\KeywordTok{mean}\NormalTok{(gNodeDegs)}
\NormalTok{n <-}\StringTok{ }\KeywordTok{length}\NormalTok{(gNodeDegs)}
\NormalTok{prob <-}\StringTok{ }\NormalTok{gDegMean }\OperatorTok{/}\StringTok{ }\NormalTok{(n}\OperatorTok{-}\DecValTok{1}\NormalTok{)}
\NormalTok{binomDegs <-}\StringTok{ }\KeywordTok{rbinom}\NormalTok{(n, n}\OperatorTok{-}\DecValTok{1}\NormalTok{, prob)}
\KeywordTok{hist}\NormalTok{(binomDegs, }\DataTypeTok{ylim =} \KeywordTok{c}\NormalTok{(}\DecValTok{0}\NormalTok{,}\DecValTok{35}\NormalTok{), }\DataTypeTok{xlim =} \KeywordTok{c}\NormalTok{(}\DecValTok{0}\NormalTok{,}\DecValTok{100}\NormalTok{), }\DataTypeTok{xlab =} \StringTok{"Degree"}\NormalTok{, }
     \DataTypeTok{main =} \StringTok{"Degree distribution of ER network of same size and mean degree of G"}\NormalTok{)}
\end{Highlighting}
\end{Shaded}

\includegraphics{Lab2_files/figure-latex/binomial-1.pdf}

G has a broad range of degree values. Variance is high, so many degree
values are not clustered around the mean. As an ER network of the same
size and mean degree has a narrower range of degree values clustered
around the mean it is not a representative model of G. It is missing
nodes of very low and very high degree, unlike G.

\subsection{scale free degree}\label{scale-free-degree}

\begin{Shaded}
\begin{Highlighting}[]
\NormalTok{k =}\StringTok{ }\DecValTok{2}\OperatorTok{:}\NormalTok{(n}\OperatorTok{-}\DecValTok{1}\NormalTok{)}
\NormalTok{unnormDensGam1 <-}\StringTok{ }\NormalTok{k}\OperatorTok{^}\NormalTok{(}\OperatorTok{-}\DecValTok{1}\NormalTok{)}
\NormalTok{unnormDensGam2 <-}\StringTok{ }\NormalTok{k}\OperatorTok{^}\NormalTok{(}\OperatorTok{-}\DecValTok{2}\NormalTok{)}
\NormalTok{unnormDensGam3 <-}\StringTok{ }\NormalTok{k}\OperatorTok{^}\NormalTok{(}\OperatorTok{-}\DecValTok{3}\NormalTok{)}

\KeywordTok{plot}\NormalTok{(k,unnormDensGam1, }\DataTypeTok{type=} \StringTok{"l"}\NormalTok{, }\DataTypeTok{ylab =} \StringTok{"unnormalized density"}\NormalTok{, }\DataTypeTok{col=}\StringTok{"red"}\NormalTok{,}
     \DataTypeTok{main =} \StringTok{"Power-law distributions"}\NormalTok{)}
\KeywordTok{lines}\NormalTok{(k, unnormDensGam2, }\DataTypeTok{col=}\StringTok{"blue"}\NormalTok{)}
\KeywordTok{lines}\NormalTok{(k, unnormDensGam3, }\DataTypeTok{col=}\StringTok{"green"}\NormalTok{)}
\KeywordTok{legend}\NormalTok{(}\StringTok{"center"}\NormalTok{, }\DataTypeTok{legend =} \KeywordTok{c}\NormalTok{(}\StringTok{"Gamma = 1"}\NormalTok{, }\StringTok{"Gamma = 2"}\NormalTok{, }\StringTok{"Gamma = 3"}\NormalTok{), }\DataTypeTok{col=} \KeywordTok{c}\NormalTok{(}\StringTok{"red"}\NormalTok{, }\StringTok{"blue"}\NormalTok{, }\StringTok{"green"}\NormalTok{), }\DataTypeTok{lty =} \KeywordTok{c}\NormalTok{(}\DecValTok{1}\NormalTok{,}\DecValTok{1}\NormalTok{,}\DecValTok{1}\NormalTok{))}
\end{Highlighting}
\end{Shaded}

\includegraphics{Lab2_files/figure-latex/scalefree-1.pdf}

Scale free networks would be poor models for G as the high frequency of
nodes with low degree is not characteristic of G. Out of the gamma
values plotted above, a scale free network with gamma of 1 would be the
best of the 3 scale free networks for modelling G as it has the higher
frequency of nodes with higher degree.

\subsection{Making 200 ER graphs}\label{making-200-er-graphs}

\begin{Shaded}
\begin{Highlighting}[]
\NormalTok{er200 <-}\StringTok{ }\KeywordTok{sapply}\NormalTok{(}\KeywordTok{rep}\NormalTok{(n, }\DecValTok{200}\NormalTok{), }\ControlFlowTok{function}\NormalTok{(x) }\KeywordTok{list}\NormalTok{(}\KeywordTok{sampleER}\NormalTok{(x, prob)))}
\NormalTok{er200Btwn <-}\StringTok{ }\KeywordTok{sapply}\NormalTok{(er200, betweenness, }\DataTypeTok{gmode =} \StringTok{"graph"}\NormalTok{, }\DataTypeTok{cmode =} \StringTok{"undirected"}\NormalTok{)}
\NormalTok{er200MeanBtwn <-}\StringTok{ }\KeywordTok{apply}\NormalTok{(er200Btwn, }\DecValTok{2}\NormalTok{, mean)}
\NormalTok{gBetween <-}\StringTok{ }\KeywordTok{betweenness}\NormalTok{(G, }\DataTypeTok{gmode =} \StringTok{"graph"}\NormalTok{, }\DataTypeTok{cmode =} \StringTok{"undirected"}\NormalTok{)}
\NormalTok{gMeanBtwn <-}\StringTok{ }\KeywordTok{mean}\NormalTok{(gBetween)}
\KeywordTok{hist}\NormalTok{(er200MeanBtwn, }\DataTypeTok{xlim =} \KeywordTok{c}\NormalTok{(}\DecValTok{29}\NormalTok{, }\DecValTok{32}\NormalTok{), }\DataTypeTok{xlab =} \StringTok{"mean betweenness"}\NormalTok{,}
     \DataTypeTok{main =} \KeywordTok{paste}\NormalTok{(}\StringTok{"Mean betweenness of ER networks of n ="}\NormalTok{, n, }\StringTok{"and p ="}\NormalTok{, }\KeywordTok{round}\NormalTok{(prob,}\DecValTok{4}\NormalTok{)))}
\KeywordTok{abline}\NormalTok{(}\DataTypeTok{v =}\NormalTok{ gMeanBtwn, }\DataTypeTok{col =} \StringTok{"red"}\NormalTok{)}
\KeywordTok{legend}\NormalTok{(}\StringTok{"topleft"}\NormalTok{, }\DataTypeTok{legend =} \KeywordTok{c}\NormalTok{(}\StringTok{"Mean betweenness of G"}\NormalTok{), }\DataTypeTok{col =} \KeywordTok{c}\NormalTok{(}\StringTok{"red"}\NormalTok{), }\DataTypeTok{lty =} \KeywordTok{c}\NormalTok{(}\DecValTok{1}\NormalTok{))}
\end{Highlighting}
\end{Shaded}

\includegraphics{Lab2_files/figure-latex/200ER Betweenness-1.pdf}

\begin{Shaded}
\begin{Highlighting}[]
\NormalTok{er200DegCen <-}\StringTok{ }\KeywordTok{sapply}\NormalTok{(er200, }\ControlFlowTok{function}\NormalTok{(x) }\KeywordTok{centralization}\NormalTok{(x, degree, }\DataTypeTok{mode =} \StringTok{"graph"}\NormalTok{))}
\NormalTok{gDegCen <-}\StringTok{ }\KeywordTok{centralization}\NormalTok{(G, degree, }\DataTypeTok{mode =} \StringTok{"graph"}\NormalTok{)}
\KeywordTok{hist}\NormalTok{(er200DegCen, }\DataTypeTok{xlab =} \StringTok{"Degree centralization"}\NormalTok{, }\DataTypeTok{xlim =} \KeywordTok{c}\NormalTok{(}\FloatTok{0.06}\NormalTok{, }\FloatTok{0.44}\NormalTok{), }
     \DataTypeTok{main =} \KeywordTok{paste}\NormalTok{(}\StringTok{"Degree centralization of ER networks of n ="}\NormalTok{, n, }\StringTok{"and p ="}\NormalTok{, }\KeywordTok{round}\NormalTok{(prob,}\DecValTok{4}\NormalTok{)))}
\KeywordTok{abline}\NormalTok{(}\DataTypeTok{v =}\NormalTok{ gDegCen, }\DataTypeTok{col =} \StringTok{"red"}\NormalTok{)}
\KeywordTok{legend}\NormalTok{(}\StringTok{"top"}\NormalTok{, }\DataTypeTok{legend =} \KeywordTok{c}\NormalTok{(}\StringTok{"Degree centralization of G"}\NormalTok{), }\DataTypeTok{col =} \KeywordTok{c}\NormalTok{(}\StringTok{"red"}\NormalTok{), }\DataTypeTok{lty =} \KeywordTok{c}\NormalTok{(}\DecValTok{1}\NormalTok{))}
\end{Highlighting}
\end{Shaded}

\includegraphics{Lab2_files/figure-latex/200ER degree centralization-1.pdf}

\begin{Shaded}
\begin{Highlighting}[]
\NormalTok{er200LCC <-}\StringTok{ }\KeywordTok{sapply}\NormalTok{(er200, localclustering)}
\NormalTok{gLCC <-}\StringTok{ }\KeywordTok{localclustering}\NormalTok{(G)}
\KeywordTok{hist}\NormalTok{(er200LCC, }\DataTypeTok{xlim =} \KeywordTok{c}\NormalTok{(}\FloatTok{0.4}\NormalTok{,}\FloatTok{0.9}\NormalTok{), }\DataTypeTok{xlab =} \StringTok{"mean local clustering coefficient"}\NormalTok{,}
     \DataTypeTok{main =} \KeywordTok{paste}\NormalTok{(}\StringTok{"Mean local clustering coefficient of }\CharTok{\textbackslash{}n}\StringTok{ER networks of n ="}\NormalTok{, n, }\StringTok{"and p ="}\NormalTok{, }\KeywordTok{round}\NormalTok{(prob,}\DecValTok{4}\NormalTok{)))}
\KeywordTok{abline}\NormalTok{(}\DataTypeTok{v =}\NormalTok{ gLCC, }\DataTypeTok{col =} \StringTok{"red"}\NormalTok{)}
\KeywordTok{legend}\NormalTok{(}\StringTok{"center"}\NormalTok{, }\DataTypeTok{legend =} \KeywordTok{c}\NormalTok{(}\StringTok{"Mean local clustering }\CharTok{\textbackslash{}n}\StringTok{coefficient of G"}\NormalTok{), }\DataTypeTok{col =} \KeywordTok{c}\NormalTok{(}\StringTok{"red"}\NormalTok{), }\DataTypeTok{lty =} \KeywordTok{c}\NormalTok{(}\DecValTok{1}\NormalTok{))}
\end{Highlighting}
\end{Shaded}

\includegraphics{Lab2_files/figure-latex/200ER local clustering coefficient-1.pdf}

Betweenness is a measure of node importance via how often it appears on
the shortest paths between pairs of nodes in the network.
\(\displaystyle C_b(u) = \sum_{t,s \in V}\frac{\sigma_{st}(u)}{\sigma_{st}}\)
\(\sigma_{st}(u)\) = number of shortest paths from node \(s\) to node
\(t\) that go through node u \(\sigma_{st}\) = total number of shortest
paths from node \(s\) to node \(t\)

Degree centralization is a measure of how much the degree of the nodes
in a network deviate from the max degree in the network (normalised to
the theoretical max degree possible).
\(\displaystyle deg^*(G) = \sum_{i\in V}|\max_{v \in V}(deg(v)) - deg(i)|\)

Local clustering coefficient of a node \(u\) is a measure of the
connectedness of the nodes in \(u\)'s neighbourhood. More formally it is
a measure of number of edges between the nodes in \(u\)'s neighbourhood
(not including edges with \(u\)) over the total number of edges possible
for the neighbourhood.

From the above histograms we can see that ER is a poor model for G. The
smaller variance in degree compared to G is reflected in the degree
centralization histogram, with the value for G falling considerably
outside the distribution of the ER values. The Mean local clustering
coefficient histogram shows that G's neighbourhoods are considerably
more connected than those of a ER network. The mean betweenness of G
being higher than the ER networks implies that there are less redundant
shortest paths in G compared to the ER networks. The difference in
values for these metrics implies G has a structure that gets lost when
randomized. Genes with a specific function only involved in a specific
pathway will be represented in G as a node with low degree. Likewise
regulatory genes like transcription factors involved in multiple
pathways would be represented as nodes of high degree. These nodes would
not occur in graphs like ER where every node has roughly the same
degree. How genes in the same pathways often regulate each other in
various feedback loops is reflected in G's higher than random local
clustering coefficient.

\subsection{Making 200 graphs sampled from the degree distribution of
G}\label{making-200-graphs-sampled-from-the-degree-distribution-of-g}

\begin{Shaded}
\begin{Highlighting}[]
\NormalTok{dd200 <-}\StringTok{ }\KeywordTok{sapply}\NormalTok{(}\KeywordTok{rep}\NormalTok{(}\DecValTok{1}\NormalTok{, }\DecValTok{200}\NormalTok{), }\ControlFlowTok{function}\NormalTok{(x) }\KeywordTok{list}\NormalTok{(}\KeywordTok{rgraphFromDegreeDist}\NormalTok{(G)))}
\NormalTok{dd200Btwn <-}\StringTok{ }\KeywordTok{sapply}\NormalTok{(dd200, betweenness, }\DataTypeTok{gmode =} \StringTok{"graph"}\NormalTok{, }\DataTypeTok{cmode =} \StringTok{"undirected"}\NormalTok{)}
\NormalTok{dd200MeanBtwn <-}\StringTok{ }\KeywordTok{apply}\NormalTok{(dd200Btwn, }\DecValTok{2}\NormalTok{, mean)}
\KeywordTok{hist}\NormalTok{(dd200MeanBtwn, }\DataTypeTok{xlim =} \KeywordTok{c}\NormalTok{(}\DecValTok{31}\NormalTok{, }\DecValTok{43}\NormalTok{), }\DataTypeTok{xlab =} \StringTok{"mean betweenness"}\NormalTok{,}
     \DataTypeTok{main =} \KeywordTok{paste}\NormalTok{(}\StringTok{"Mean betweenness of networks sampled from degree distribution of G"}\NormalTok{))}
\KeywordTok{abline}\NormalTok{(}\DataTypeTok{v =}\NormalTok{ gMeanBtwn, }\DataTypeTok{col =} \StringTok{"red"}\NormalTok{)}
\KeywordTok{legend}\NormalTok{(}\StringTok{"topleft"}\NormalTok{, }\DataTypeTok{legend =} \KeywordTok{c}\NormalTok{(}\StringTok{"Mean betweenness of G"}\NormalTok{), }\DataTypeTok{col =} \KeywordTok{c}\NormalTok{(}\StringTok{"red"}\NormalTok{), }\DataTypeTok{lty =} \KeywordTok{c}\NormalTok{(}\DecValTok{1}\NormalTok{))}
\end{Highlighting}
\end{Shaded}

\includegraphics{Lab2_files/figure-latex/200dd Betweenness-1.pdf}

\begin{Shaded}
\begin{Highlighting}[]
\NormalTok{dd200DegCen <-}\StringTok{ }\KeywordTok{sapply}\NormalTok{(dd200, }\ControlFlowTok{function}\NormalTok{(x) }\KeywordTok{centralization}\NormalTok{(x, degree, }\DataTypeTok{mode =} \StringTok{"graph"}\NormalTok{))}
\KeywordTok{hist}\NormalTok{(dd200DegCen, }\DataTypeTok{xlab =} \StringTok{"Degree centralization"}\NormalTok{,}
     \DataTypeTok{main =} \KeywordTok{paste}\NormalTok{(}\StringTok{"Degree centralization of networks }\CharTok{\textbackslash{}n}\StringTok{sampled from degree distribution of G"}\NormalTok{))}
\KeywordTok{abline}\NormalTok{(}\DataTypeTok{v =}\NormalTok{ gDegCen, }\DataTypeTok{col =} \StringTok{"red"}\NormalTok{)}
\KeywordTok{legend}\NormalTok{(}\FloatTok{0.367}\NormalTok{, }\DecValTok{40}\NormalTok{, }\DataTypeTok{legend =} \KeywordTok{c}\NormalTok{(}\StringTok{"Degree centralization of G"}\NormalTok{), }\DataTypeTok{col =} \KeywordTok{c}\NormalTok{(}\StringTok{"red"}\NormalTok{), }\DataTypeTok{lty =} \KeywordTok{c}\NormalTok{(}\DecValTok{1}\NormalTok{))}
\end{Highlighting}
\end{Shaded}

\includegraphics{Lab2_files/figure-latex/200dd degree centralizaion-1.pdf}

\begin{Shaded}
\begin{Highlighting}[]
\NormalTok{dd200LCC <-}\StringTok{ }\KeywordTok{sapply}\NormalTok{(dd200, localclustering)}
\KeywordTok{hist}\NormalTok{(dd200LCC, }\DataTypeTok{xlim =} \KeywordTok{c}\NormalTok{(}\FloatTok{0.5}\NormalTok{,}\FloatTok{0.85}\NormalTok{), }\DataTypeTok{xlab =} \StringTok{"Mean Local Clustering Coefficient"}\NormalTok{,}
     \DataTypeTok{main =} \StringTok{"Mean local Clustering Coefficient of }\CharTok{\textbackslash{}n}\StringTok{networks sampled from degree distribution of G"}\NormalTok{)}
\KeywordTok{abline}\NormalTok{(}\DataTypeTok{v =}\NormalTok{ gLCC, }\DataTypeTok{col =} \StringTok{"red"}\NormalTok{)}
\KeywordTok{legend}\NormalTok{(}\FloatTok{0.66}\NormalTok{, }\DecValTok{50}\NormalTok{, }\DataTypeTok{legend =} \KeywordTok{c}\NormalTok{(}\StringTok{"Mean local clustering }\CharTok{\textbackslash{}n}\StringTok{coefficient of G"}\NormalTok{), }\DataTypeTok{col =} \KeywordTok{c}\NormalTok{(}\StringTok{"red"}\NormalTok{), }\DataTypeTok{lty =} \KeywordTok{c}\NormalTok{(}\DecValTok{1}\NormalTok{))}
\end{Highlighting}
\end{Shaded}

\includegraphics{Lab2_files/figure-latex/200dd local clustering coefficient-1.pdf}

Building a network by sampling from the degree distribution of G
provides a model that is at least better than ER, but still looses the
structures present in G. Unlike ER, the mean betweenness of these random
graphs are all higher than that of G. This is possibly due to how G,
being a natural network, would likely have intraconnected neigbourhoods
representing specific pathways and a few nodes that connect the
neighbourhoods to each other. The neigbourhood connecting nodes would
have high betweenness but the mean value would be reduced by nodes in
the neigbourhoods. This is because for a node in a neighbourhood \(v\),
it would be unlikely to be on the shortest path between a pair of nodes
where neither node are in \(v\)'s neighbourhood. As a random graph would
not likely share this structure nodes would more likely be on shortest
paths between other nodes. The degree centralization is centered around
the degree centralization of G which is to be expected when sampling
from the same degree distribution. The random graphs would have max
degree values similar to that of G with the other degree values
deviating a similar amount. Just like with ER networks, the mean local
clustering coefficient are higher than these random graphs, also
potentially indicating a `connected intraconnected neigbourhoods'
structure. Random graphs without a bias towards connected neighbourhoods
would have lower local clustering coefficient.

\subsection{If we had a good model for
G}\label{if-we-had-a-good-model-for-g}

If the above historgrams were to be plotted for a good model of G, the
values would be distributed around a mean equal to the values for G. A
possible model to try for G would be one that, when building the network
edge by edge, the probability of an edge between node \(s\) and \(t\)
would increase depending on the size of the intersect between nodes in
the neighbourhoods of \(s\) and \(t\). This would increase the average
local clustering coefficient, hopefully better reflecting G.


\end{document}
